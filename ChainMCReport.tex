\documentclass{article}
\usepackage{amsmath}
\usepackage{float}
\usepackage{amssymb}
\usepackage{graphicx}
\usepackage[margin=1in]{geometry}
\usepackage{multicol,caption,listings}
\usepackage{cite}

\bibliographystyle{plain}

\begin{document}

\title{Polymer Chain Monte Carlo}
\author{A. Lovell}
\maketitle

\noindent \textbf{Abstract}  Write the abstract here - give a brief overview about what you have done and what you're planning on showing.

\begin{multicols}{2}

\section{Introduction}

\section{Theory}

\subsection{Interaction}

\subsection{Algorithm}

Maybe this will be its own section instead of a subsection.\\

The algorithm that we use for this Monte Carlo simulation is the Rosenbluth algorithm.  

\section{Initial Conditions}

I don't know if this will actually be relevant.  

\section{Discussion}

\section{Conclusion}

Talk briefly about what we calculated.  \textbf{Varying size, varying dimension, varying temperatures - find the critical temperature Theta.}\\

There is always future work that could be done regarding this project.  While we calculated a polymer chain in three-dimensions, it would also be interesting to calculate higher dimensional chains and examine the scaling of the size of the chain with the number of atoms added to the chain.  The same results as \textbf{(put reference to either equations or figures)} should hold, but it would be an interesting project nonetheless to generalize the code.  It would also be interesting to examine the difference between this free self-avoiding walk calculated here and a self-avoiding walk confined to a grid.  In this way, we could see how the size of the polymer is related to the number of atoms and the dimensionality of the problem.  \\

There are also several thermodynamical quantities that could be calculated from this system, including specific heat, thermal energy, and \textbf{many more}.  Still, we successfully modeled a polymer chain of \textbf{N} atoms in \textbf{two-} and three-dimensions.  

\end{multicols}

\bibliography{ChainBib}

\end{document}