\documentclass{article}
\usepackage{amsmath}
\usepackage{float}
\usepackage{amssymb}
\usepackage{graphicx}
\usepackage[margin=1in]{geometry}
\usepackage{multicol,caption,listings}
\usepackage{cite}

\bibliographystyle{plain}

\begin{document}

\title{Polymer Chain Monte Carlo}
\author{A. Lovell}
\maketitle

\noindent \textbf{Abstract}  Write the abstract here - give a brief overview about what you have done and what you're planning on showing.

\begin{multicols}{2}

\section{Introduction}

Polymer are chains atoms or molecules of the same type that interact directly between sequential atoms through a spring force and through other pairs via some other long- or short-range potential (given, for example, by hydrogen bonds, van der Waals interactions, or the Lennard-Jones interaction).  The typical number of atoms in a polymer is between $10^3$ and $10^5$.  \\

One would often think that a molecular dynamics calculation would be used to simulate a polymer chain.  In this case, the movement of the chain would be calculated at each time step, the forces between each molecule calculated, and then used to update the chain's next position.  However, because of the different types of motion involved in the problem - ranging from fast motion of the individual units to slow motion of the chain as a whole - this process can be inefficient, or even impossible, for long chains.  \cite{PhilNotes} Instead, Monte Carlo simulations are used to discover properties of the polymer.  \\

This report is broken up into the following sections.  In Section \ref{theory}, we will discuss the theory behind the problem, including the interaction between atoms and the algorithm that was used to run the Monte Carlo simulation.  Section \ref{IC} gives a brief discussion of the initial conditions of the simulation and the different grids that were explored.  The results of the simulation are discussed in Section \ref{discuss}.  Finally, in Section \ref{concl}, we provide a summary of our work along with progress that could be make to this project in the future. \\

\section{Theory}
\label{theory}

\subsection{Interaction}

Polymer chains can be modeled in several ways.  One of the most common ways is describing the interaction between sequential pairs of atoms as a stiff spring and putting in a Lennard-Jones potential between every other pair of atoms.  \textbf{Reference this.}  However, in this project we will take a somewhat simplified approach.  Instead of stiff springs, we will fix the length between each successive pair of atoms, while keeping the interaction between all other pairs a Lennard-Jones interaction.  There are, of course, more complicated that could be implemented to fold the polymer into more complex shapes (such as proteins or even origami ducks), but those will not be explored here.\\

The Lennard-Jones interaction is 

\begin{equation}
V_{LJ}(r) = 4\epsilon \left [ \left ( \frac{\sigma}{r} \right ) ^{12} - \left ( \frac{\sigma}{r}) \right ) ^6 \right ]
\end{equation}

\noindent where $\epsilon$ is the depth of the interaction, $\sigma$ describes the particle size, and $r$ is the distance between the interacting pairs.  This form takes into account the short-range repulsive force between the two atoms due to Coulomb repulsion and the longer-range attractive forces from dipole-dipole and dipole-induced dipole forces.  An example of the Lennard-Jones potential can be found in Figure \textbf{(or other)}.

\subsection{Algorithm}

Maybe this will be its own section instead of a subsection.\\

The algorithm that we use for this Monte Carlo simulation is the Rosenbluth algorithm.  

\section{Initial Conditions}
\label{IC}

I don't know if this will actually be relevant.  

\section{Discussion}
\label{discuss}

\section{Conclusion}
\label{concl}

Talk briefly about what we calculated.  \textbf{Varying size, varying dimension, varying temperatures - find the critical temperature Theta.}\\

There is always future work that could be done regarding this project.  While we calculated a polymer chain in three-dimensions, it would also be interesting to calculate higher dimensional chains and examine the scaling of the size of the chain with the number of atoms added to the chain.  The same results as \textbf{(put reference to either equations or figures)} should hold, but it would be an interesting project nonetheless to generalize the code.  It would also be interesting to examine the difference between this free self-avoiding walk calculated here and a self-avoiding walk confined to a grid.  In this way, we could see how the size of the polymer is related to the number of atoms and the dimensionality of the problem.  \\

There are also several thermodynamical quantities that could be calculated from this system, including specific heat, thermal energy, and \textbf{many more}.  Although we used a fixed bond length between each successive atom and a Lennard Jones interaction between all pairs of atoms, there are many other interactions that could describe a more realistic polymer.  This includes using a stiff spring to model the interaction between successive atoms.  We also could have changed the strength of the interaction between various pairs to see what kinds of shapes we could form with our polymer chain.  Still, we successfully modeled a polymer chain of \textbf{N} atoms in \textbf{two-} and three-dimensions.  

\end{multicols}

\bibliography{chainbib}

\end{document}